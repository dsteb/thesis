\chapter{Conclusion and future work}
\label{conclusion}

\section{Conclusion}
In this work it was implemented the Executor component for the autonomic system based with the adaption based on the MAPE framework. It were considered different virtualization types: coarse-grained the classic one (virtual machines) and the fine-grained container virtualization. During the implementing and experiments we can see advantages of container virtualization: using the fine-grained adaptation capabilities can greatly improve performance when autoscaling cloud-based web-application.

Modern cloud micro services and multitier architectures can benefit from using the autoscaling techniques, providing the decrease in the computational resource consuming, while showing high performance at the same time. The decreasing in the resource consumption it is not only sustainable approach to the environment, but also the reduction of the expenses considering the pay-as-you-go services like AWS.


\section{Future work}
As it is said in the conclusion part the adaptation using containers can greatly improve the performance of the autoscaling. As the evaluation part was not the part of this paper, the evaluation and the proof should be done as the future work.

Also as the future work it is considered the support of other infrastructure or cloud engines: Google App Engine or Microsoft Azure Cloud

Also the future work comprises the integration of feature adaptation by extending the adaptation hook mechanism, an extension of the planner to make it work hierarchically with respect to the controlled resources, and even finer-grained solution to control the CPUs cores allocated to a container, and further evaluation on more case studies of different kinds.