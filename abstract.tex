\newpage
\chapter*{Abstract}

\addcontentsline{toc}{chapter}{Abstract}

Nowadays Modern Web applications are often deployed and executed on a Cloud infrastructure which provides a convenient on-demand approach for renting resources and easy-to-use horizontal scaling capabilities. The workload of Web applications is continuously changing over time and unexpected peaks of requests can happen, making the system unable to respond. For this reason the autonomic adaptation is an emerging solution to automatically adapt the resources allocated to the application according to the incoming traffic, cpu-utilization, and other metrics. Our ongoing autonomic initiative is based on the MAPE architecture (Monitor-Analyze-Plan-Execute). This thesis focuses on the Execute component. 

While the state of the art solutions focus on adjusting the number of Virtual Machines allocated to the application, the containerization, a novel kind of virtualization that takes place at the operating system level, is emerging and is becoming popular. Containers are linux processes that can run sandboxed on a shared host operating system. This means that each container does not contain an entire operating system making this technology more lightweight and faster to boot comparing to Virtual Machines.

The contribution of this thesis is the implementation of the Execute component that exploits the usage of both Virtual Machines and containers enabling a faster and finer-grained adaptation and multi-layer adaptation. We consider not only the adaptation at the infrastructure layer, but we also adapt the middleware software that enables the execution of application specific code as application servers, DBMS and so on. We have implemented two approaches for the "Execute" component: the monolithic and the hierarchical one. The former consists of a centralized architecture where only monitoring sensors are distributed among the nodes, the latter consists in completely distributed architecture where all the MAPE components are replicated at each level of the hierarchy (container, VM, cluster of VMs, etc.).

\ifx
Modern Web applications are now often built on the Cloud infrastructure which provides the flexibility and the scalability for applications. Cloud solutions can be considered as environmentally friendly, as they can reduce the usage of the infrastructure resources, hence the electricity consumption and costs. The workload on the modern Web applications is continuously changing and we can exploit it to adapt applications by means of the resource allocation (Virtual Machines).

The implementation of our adaptation solution is based on the MAPE approach (Monitor-Analyze-Plan-Execute). While the Monitor-Analyze-Plan parts are out of the focus of this thesis, the work focuses on the Execute part.

The contributions of this thesis are to use the coarse-grained  and fine-grained adaptations (Virtual Machines and containers) and also the multilayer adaptation. We consider not only the infrastructure, but also the middle-ware adaptations: application servers, DBMS and so on.
We have implemented two approaches for the "Execute" component: the monolithic and the hierarchical one. The scope of this thesis is the implementation details of both these approaches.

----

The world is moving to economy of the nature resources and environment-friendly solutions that can be called sustainable solutions. Approach of sustainability can be also applied to large computer infrastructures to adapt allocated resources depending on the load, on the time of the day or on other input.

Modern Web applications exploit Cloud infrastructures to
scale their resources and cope with sudden changes in the
workload. While the state of practice focuses on dynamically adding and removing virtual machines, we also implement and look on more fine-grained solution: operation system-level containerization.

In this thesis we present an autoscaling technique that allows containerized applications to scale their resources both at the VM level and at the container level. Furthermore, applications can combine this infrastructural adaptation with platform-level adaptation. The autoscaling is made possible by our planner, which consists of a discrete-time feedback
controller.

The work investigates coarse-grained virtualization techniques like virtual machine comparing to light-weight fine-grained techniques like operating system-level virtualization (containers virtualization). The scope of the thesis is implementation of both techniques and comparing the advantages and disadvantages of both of them.

The work has been validated using two application bench-
marks deployed to Amazon EC2. Our experiments show
that our planner outperforms Amazon’s AutoScaling by 78%
on average without containers; and that the introduction of
containers allows us to improve by yet another 46% on av-
erage.

----

This paper presents the research on Autonomic Systems with different types of virtualization. It were investigated coarse-grained virtualization techniques like virtual machines or cloud instances, and light-weight fine-grained techniques like operating system-level virtualization (containers). The thesis scope is implementation case of Autonomic System using different virtualization elements and considering the advantages and disadvantages of using containers for building Autonomic Systems.
\fi